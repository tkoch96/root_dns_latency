\documentclass[sigconf,nonacm,10pt]{acmart}
\usepackage{lmodern}
\usepackage{amssymb,amsmath}
\usepackage{ifxetex,ifluatex}
\usepackage{fixltx2e} % provides \textsubscript
\ifnum 0\ifxetex 1\fi\ifluatex 1\fi=0 % if pdftex
  \usepackage[T1]{fontenc}
  \usepackage[utf8]{inputenc}
\else % if luatex or xelatex
  \ifxetex
    \usepackage{mathspec}
  \else
    \usepackage{fontspec}
  \fi
  \defaultfontfeatures{Ligatures=TeX,Scale=MatchLowercase}
  \newcommand{\euro}{€}
\fi
% use upquote if available, for straight quotes in verbatim environments
\IfFileExists{upquote.sty}{\usepackage{upquote}}{}
% use microtype if available
\IfFileExists{microtype.sty}{%
\usepackage{microtype}
\UseMicrotypeSet[protrusion]{basicmath} % disable protrusion for tt fonts
}{}
\usepackage{hyperref}
\PassOptionsToPackage{usenames,dvipsnames}{color} % color is loaded by hyperref
\hypersetup{unicode=true,
            pdftitle={If a Path is Inflated, and Noone Uses It, Is It Inefficient{[}a{]}?},
            colorlinks=true,
            linkcolor=blue,
            citecolor=blue,
            anchorcolor=blue,
            urlcolor=blue,
            breaklinks=true}
\urlstyle{same}  % don't use monospace font for urls
\usepackage{natbib}
\bibliographystyle{ACM-Reference-Format.bst}
\usepackage{graphicx,grffile}
\makeatletter
\def\maxwidth{\ifdim\Gin@nat@width>\linewidth\linewidth\else\Gin@nat@width\fi}
\def\maxheight{\ifdim\Gin@nat@height>\textheight\textheight\else\Gin@nat@height\fi}
\makeatother
% Scale images if necessary, so that they will not overflow the page
% margins by default, and it is still possible to overwrite the defaults
% using explicit options in \includegraphics[width, height, ...]{}
\setkeys{Gin}{width=\maxwidth,height=\maxheight,keepaspectratio}
\setlength{\emergencystretch}{3em}  % prevent overfull lines
\providecommand{\tightlist}{%
  \setlength{\itemsep}{0pt}\setlength{\parskip}{0pt}}
\setcounter{secnumdepth}{5}

%\usepackage{caption}
%\renewcommand{\captionfont}{\small} %small fonts for caption
%\renewcommand{\captionlabelfont}{\small}

% \usepackage{url}
% \usepackage{balance}

%% Adding URL breaks
% \makeatletter
% \g@addto@macro{\UrlBreaks}{\UrlOrds}
% \makeatother

% \usepackage{lastpage}

%\usepackage[aboveskip=2pt]{subcaption} %for subfigures

%\setlength{\textfloatsep}{0pt} %spacing between figures and texts
%\setlength{\floatsep}{0pt} 
%\setlength{\dblfloatsep}{0pt}
%\setlength{\dbltextfloatsep}{0pt}
%\setlength{\abovecaptionskip}{0pt}
%\renewcommand{\footnotesize}{\scriptsize}

%\usepackage{etoolbox} % spacing between formula and text
%\apptocmd\normalsize{%
%\abovedisplayskip=0pt
%\abovedisplayshortskip=0pt
%\belowdisplayskip=0pt
%\belowdisplayshortskip=0pt
%}{}{}

%\let\oldfootnote\footnote %small footnote
%\renewcommand{\footnote}[1]{{\oldfootnote{\scriptsize #1}}}


\usepackage{multirow}
\usepackage{graphicx}
%% conference information
% \acmYear{2019}
% \copyrightyear{2019}
% \acmConference{CoNEXT '19}{December 9-12, 2019}{Orlando, Florida, USA}

\hyphenation{name-spaces}

%% System name
\newcommand\thepeering{\textsc{Peering}\xspace}
\newcommand\peering{\textsc{Peering}\xspace}
%\newcommand\peering{\textsc{BGPPlatform}\xspace}
\newcommand\vbgp{\texttt{vBGP}\xspace}
\newcommand\testbed{platform\xspace}
\newcommand\testbeds{platforms\xspace}

%%% Macros for convenience
%% References
\newcommand{\secref}[1]{\S\ref{sec:#1}}
\newcommand{\figref}[1]{Figure~\ref{fig:#1}}
\newcommand{\tabref}[1]{Table~\ref{tab:#1}}
%% to refer to lines in graphs
\newcommand{\linename}[1]{\emph{#1}\xspace}
\newcommand{\parab}[1]{\smallskip\noindent {\bf #1}}

%% Common latin terms
\newcommand{\etc}{\emph{etc.}\xspace}
\newcommand{\ie}{\emph{i.e.,}\xspace}
\newcommand{\eg}{\emph{e.g.,}\xspace}
\newcommand{\etal}{\emph{et al.}\xspace}
\newcommand{\aka}{a.k.a\xspace}

%% editing notes
%\newcommand\ekb[1]{{\color{blue}[ekb: #1]}}
\newcommand\tbd[1]{{\color{red}{\bf TBD: #1}}}
%\newcommand\new[1]{#1}
%\newcommand\cut[1]{}
\definecolor{orange}{rgb}{1.0, 0.31, 0.0}
%\newcommand\edit[2]{#2}
%\newcommand\new[1]{{#1}}
%\newcommand\cut[1]{{#1}}

% terms
\newcommand\noescape[1]{#1}

\newcommand{\rightdownarrow}{\mathrel{\scalebox{1}[-1]{$\Rsh$}}}
\newcommand{\cmark}{\color{green}\ding{51}}
\newcommand{\xmark}{\color{red}\ding{55}}
\newcommand{\metroas}{$\left<\texttt{metro, AS, region}\right>$\xspace}
\newcommand{\fe}{front-end\xspace}
\newcommand{\feplural}{front-ends\xspace}
\newcommand{\capfe}{Front-end\xspace}
\newcommand{\capfeplural}{Front-ends\xspace}

\NewDocumentCommand{\rotseventy}{O{70} O{1em} m}{\makebox[#2][l]{\rotatebox{#1}{#3}}}


% De-Anonymized Commands
\iffalse
\newcommand\ISI{\textsc{the Information Sciences Institute (ISI) at USC}\xspace}
\fi

% Anonymized Commands
\newcommand\ISIone{a research lab in a university in the United States\xspace} % introductory reference
\newcommand\ISItwo{the research lab\xspace} % more general references
\newcommand\ISIthree{a research lab\xspace}

\title{If a Path is Inflated, and Noone Uses It, Is It Inefficient{[}a{]}?}
\author{
            Thomas Koch (Columbia University)
         \and 
            Matt Calder (Microsoft Research)
         \and 
            Ethan Katz-Bassett (Columbia University)
         \and 
            John Heidemann (ISI)
         \and 
            Arpit Gupta (UCSB)
        }
\date{}
\pagestyle{plain}

\begin{document}
\maketitle

\iffalse

subtitle: Paper \#4, \pageref{endofconclusionlabel} pages
(\ref{TotPages} with citations)

classoption: - natbib=true - table header-includes: -
\renewcommand{\shortauthors}{Anonymized} -
\renewcommand{\shorttitle}{User-Perceived Root DNS Latency} -
\setcopyright{none} -
\settopmatter{printacmref=false, printccs=false, printfolios=true} -
\acmDOI{} - \acmISBN{} ---

\fi

\iffalse

Github: https://github.com/tkoch96/root\_dns\_latency Table of contents

\section*{Abstract}\label{abstract}
\addcontentsline{toc}{section}{Abstract}

\section{Introduction}\label{introduction}

\section{DNS -- A Practical Viewpoint}\label{dns-a-practical-viewpoint}

\subsection{Great Latency Savings
Potential}\label{great-latency-savings-potential}

\subsection{Practical DNS Pitfalls}\label{practical-dns-pitfalls}

\section{A Close Look at a Recursive
Resolver}\label{a-close-look-at-a-recursive-resolver}

\section{A Global Look at Users of a Large
CDN}\label{a-global-look-at-users-of-a-large-cdn}

\subsection{Data Sources and
Processing}\label{data-sources-and-processing}

\subsection{Root Latency Experienced by
Users}\label{root-latency-experienced-by-users}

\section{Related Work}\label{related-work}

\section{Conclusion}\label{conclusion}

\fi

\iffalse

Anycast is a means of distributing content that has been praised for its
simplicity and performance, yet criticized for inflating user latencies
in some cases. The r{[}b{]}oot DNS servers are frequent sources of
information for studies analyzing anycast, since the information is
relatively easy to obtain. We argue that the root DNS servers are not
ideal test subjects for studies either offering criticism of or
suggesting improvements for anycast, since users rarely interact with
the root DNS infrastructure. That is, we question whether anycast's
inefficiencies are truly inefficiencies if they have no real, measurable
effect.\break
We demonstrate that simple caching policies of recursive resolvers limit
a users' exposure to the root DNS infrastructure, and quantify globally
how much latency users experience each day due to root DNS resolution.
Our results indicate most users spend no more than 15 ms each day
waiting for root DNS query resolution, and that path inflation caused by
anycast results in an additional 2 ms for most users -- latencies that
are hardly distinguishable from noise. These findings therefore indicate
that future studies should be more careful when drawing from root DNS
server data to support arguments regarding anycast latency inflation.

\fi

\iffalse

Link to outline \fi

\section*{Abstract}\label{abstract-1}
\addcontentsline{toc}{section}{Abstract}

Anycast is a means of distributing content that has been praised for its
simplicity and performance, yet criticized for inflating user latencies
in some cases. However, anycast CDNs and anycasted DNS resolvers serve
latency-sensitive content to millions presenting an apparent
contradiction. To resolve this discrepancy in understanding, we analyze
anycast in two important settings: the root DNS and a large anycast CDN,
placing an emphasis on how anycast's performance translates to end user
latency. \break
First, we question whether latency matters at all, and find that while
it does in the CDN setting, it makes little difference in the root DNS
due to heavy caching near end users -- accounting for about 2 ms per day
for users. We then examine how anycast specifically impacts performance
to end users, and again demonstrate users see negligible effects from
anycast path inflation. Users of the anycast CDN, conversely, can
experience a few ms per page load of anycast path inflation, with the
effect becoming more prevalent in larger anycast deployments. The
effects of the inflation, however, are dwarfed by the performance
benefits users see by, on average, visiting closer anycast sites. Hence
we demonstrate that although anycast can inflate latencies, the degree
to which this affects users largely depends on the setting, and the
prevalence of this inflation does not tell the whole story -- anycast
still does a good job routing users to sites.

\section{Introduction}\label{introduction-1}

\label{sec:introduction} IP anycast, a system in which geographically
diverse servers known as anycast replicas all use the same IP address,
is an important part of a number of operational DNS
\cite{root_servers, cloudflare_anycast, akamai_anycast, route53_anycast, google_public_dns}
and CDN \cite{calder2015analyzing,edgecast_anycast,amazon_cloudfront}
systems today, in part because of its support to improve latency to
clients and decrease load on each anycast server
\cite{katabi2000framework,metz2002ip,rfc_1546}. However, numerous
studies have argued that anycast provides sub-optimal performance for
some users, compared to the lowest latency anycast could offer in
theory. Despite these implications, how these inefficiencies impact user
experience is not well understood. Anycast, therefore, presents a
paradox providing benefits of increased capacity and decreased latency
while also purportedly hurting performance. To resolve this, we take a
step back and ask pointed questions -- why do distributed systems such
as CDNs and operational DNS use anycast, how does anycast affect their
performance, and how does this performance ultimately translate to end
user experience? \break \break

Towards answering these questions we analyze anycast from two different
angles: the root DNS and a large anycast CDN, chosen for their
overlapping, yet distinctive, goals. The root DNS servers feature in
studies involving anycast because it is relatively easy to gain access
to root DNS data, because it is straightforward to gain access to
information about their deployments, and because they are run by several
organizations \cite{root_servers}. This last fact manifests itself in a
diverse set of deployment strategies, for essentially the same service.
Their use here thus provides us with, in addition to the above mentioned
benefits, comparisons to prior work. Examining the root DNS and anycast
CDNs is interesting because this analysis illustrates how the setting in
which we study anycast heavily influences the conclusions we can draw --
while we find that mitigating anycast path inflation is quite important
for anycast CDNs, the impact of latency \textit{at all} in the root DNS
setting is negligible. \break \break
Although we do wish to examine how anycast affects performance both at
the roots and in anycast CDNs, we first take a step back and examine
whether performance (that is, latency) matters \textit{at all}
(\autoref{sec:root_dns_latency}). We show that, from a variety of
perspectives, the effect of root DNS latency on user-perceived latency
is negligible, accounting for perhaps a few milliseconds of wait time
per day or fractions of a percent per page load. This is due mostly to
heavy caching of root DNS records. Conversely, we show that latency
matters considerably for anycast CDNs, comprising tens of percents of
page load time (\autoref{sec:does_anycast_matter_cdn}). \break
With these basic results about latency, we revisit frequently posed
questions about how anycast specifically impacts these services. We find
that, even though round-trip times differ significantly by root DNS
anycast deployment size, these differences are negligible when looked at
from a per-page load perspective, making at most an INSERT\_NUMBER
millisecond difference. Similarly, even though we find that increasing
deployment size can lead to more inflation in the roots, this inflation
negligibly factors into page load times
(\autoref{sec:root_dns_anycast}). Conversely, we find that for an
anycast CDN, although increasing deployment sizes does make anycast path
inflation more prevalent, the latency per page load decreases by tens of
milliseconds (\autoref{sec:cdn_anycast}) with additional sites.
Moreover, regardless of deployment size, the path inflation for the
anycast CDN is less than half that of the roots. Hence larger deployment
sizes can provide tangible latency benefits to anycast CDNs, but
probably provide little benefit in the root DNS setting, and the
magnitude of these benefits are largely dependent on deployment details.

\section{Anycast in Distributed
Systems}\label{anycast-in-distributed-systems}

\label{sec:anycast_distributed_systems}

\subsection{Potential Benefits of
Anycast}\label{potential-benefits-of-anycast}

{[} why distributed systems might use anycast, different performance
benefits they might shoot for{]}

\subsection{Root DNS anycast}\label{root-dns-anycast}

{[} brief description of dns resolution process, caching potential {]}

\subsection{CDN Anycast}\label{cdn-anycast}

{[} description of microsoft's CDN, with a focus on intricacies of
rings, and how we use them to simulate deployments{]}

\section{Does Root Latency Matter?}\label{does-root-latency-matter}

\label{sec:root_dns_latency}

{[}ISI results -- root dns latency cdf and PLT implications Main idea is
cache hit rate is high, and root requests are rarely generated. {]}

{[}DITL -- root DNS per day latency; Main idea is that amortizing
requests seen to the roots over large user populations makes latency
implications small{]}

{[}Embedded in the above is a comparison between high latency and low
latency roots with implications to user latency, so No, latency doesn't
matter{]}

\section{Anycast in the Root DNS}\label{anycast-in-the-root-dns}

\label{sec:root_dns_anycast} {[}anycast path inflation per RTT \& per
page load; Main idea is path inflation does become more prevalent with
increasing deployment, but makes no difference to users{]}{[}bar graph
showing which sites are hit, corroborates the idea that inefficiency
grows with deployment size, usually{]}

{[}use these results to suggest sites are added to root deployments for
resilience, since there is no difference from a PPL perspective{]}

\section{Does Anycast Performance Matter for
CDNs?}\label{does-anycast-performance-matter-for-cdns}

\label{sec:does_anycast_matter_cdn} {[}latency per RTT and per page load
at various ring sizes; Main idea is RTT latency (and perceived latency)
decreases when you add more anycast sites{]}{[}compared to root DNS per
page load; idea is CDN latency is orders of magnitude more significant
PPL{]}

\section{Anycast Performance in CDNs}\label{anycast-performance-in-cdns}

\label{sec:cdn_anycast} {[}path inflation per RTT/page load, by ring;
Main idea is that inflation becomes more prevalent, but latency PPL
still goes down{]}{[}inefficiency by ring; shows fewer users go to the
closest site, but still latency PPL goes down{]} {[}geographic path
inflation per RTT/page load, compared to the roots; CDN inflation
\textless{} root inflation,{[}c{]} argue that CDN works to control it
via peering{]}{[}case studies of intermetro variability, or unexpectedly
poor performance, highlighting the intricacies (time permitting){]}

\section{Related Work}\label{related-work-1}

\label{sec:related} IP anycast performance is usually studied in the
context of two applications: the root DNS servers, and CDNs. In addition
to these topics, we discuss studies of popular recursive resolvers, and
user-centric measurements of web performance.

\subsection{Root DNS Anycast
Performance}\label{root-dns-anycast-performance}

The performance of anycast in the context of root DNS is generally
gauged by anycast's ability to balance load among server replicas or
provide low latency to users. Generally, all studies conclude that
anycast successfully balances load, while latency performance depends on
the specific deployment configuration. \cite{moura2016anycast} looks at
a DDoS attack on the root name server infrastructure, and generally
shows that anycast is a good defense mechanism against such attacks. An
earlier study, \cite{sarat2006use} confirms that anycast protects the
root DNS infrastructure against such attacks and, furthermore, that
anycast routes users to an optimal location in most cases.
\cite{de2017anycast} looks at user latency to C, F, K, and L-root and
attributes better performance to good geographic location and peering
strategies. These findings coincide with an earlier study,
\cite{ballani2006measurement}, who conclude the performance of anycast
is intrinsically linked to deployment strategy. Additionally
\cite{de2017anycast} finds that as few as 12 sites can provide ``good''
latency to users. \cite{li_levin_spring_bhattacharjee_2018},
\cite{colitti2006evaluating}, \cite{de2017anycast}, and
\cite{liang2013measuring} are all examples of studies who quantify
latencies to various root servers, and note how these compare to the
(optimal) latency of the closest unicast alternative for the user who
issued the query.

\subsection{CDN Anycast Performance}\label{cdn-anycast-performance}

Some CDNs (e.g.~Cloudflare, Edgecast, Fastly) use IP anycast to augment
their serving infrastructure. When deploying an Anycast CDN (ACDN),
delivering content to users with low latency becomes a high priority, as
there is a large financial incentive to do so. The simplicity of IP
anycast comes at the cost of having coarse grained control over where
user queries land. Shifting user load between nodes during peak hours,
for example, is a challenging problem. As a potential solution,
\cite{alzoubi2011practical} and \cite{flavel2015fastroute} use DNS
redirects at ADNS servers to shift load among anycast nodes, albeit in
slightly different ways. \cite{calder2015analyzing} analyzes what
latency users are achieving, compared to optimal, when being routed to
anycast nodes and finds that 10\% of users experience a latency
inflation of at least 100 ms.

\subsection{Recursive Resolvers and the Benefits of
Caching}\label{recursive-resolvers-and-the-benefits-of-caching}

Similar to the RR analysis conducted here, \cite{jung2002dns} looks at
DNS traffic on a small network and notably finds that 16\% of queries
resulted in queries to the root, most of which were for invalid domains.
As this study is quite old, it is no surprise that this rate has
decreased (recall we observed .5\% of queries resulted in queries to the
root) since browser designers and network engineers understand the
importance of caching. \cite{callahan2013modern} also looks at a RR and
analyzes statistics of DNS exchanges occurring over it including DNS
transaction latencies. Both \cite{yu2012authority} and \cite{lentz2013d}
look at certain pathological behaviors of popular recursive resolvers,
and the implications these behaviors have on root DNS load.

\subsection{Web Performance}\label{web-performance}

Although we were unable to find any specific study that looked at how
web performance and root DNS latency were related, there are certainly
studies characterizing web performance. \cite{sundaresan2013web}
characterizes web performance bottlenecks in (at the time) new broadband
networks, and finds that latency is the main bottleneck for PLT when the
user's bandwidth exceeds 16 Mbps. However, the study does not
realistically emulate a page load and, in particular, can not analyze
the effect of having multiple DNS resolutions per page. Similarly,
\cite{asrese2016wepr} analyzes how each step of a page load contributes
to the aggregate PLT using a tool designed in-house. However, unlike
\cite{sundaresan2013web}, they did not conduct a large measurement
campaign and do not include information about multiple DNS lookups per
page. A more recent study, \cite{enghardt2019web} provides a brief
survey of web performance measurement studies and explains why it is
difficult (with current practices) to compare two different studies in
web performance.

\iffalse

(studies looking at anycast in context of root DNS servers) Anycast
Performance, Problems and Potential how many sites are enough? A
measurement based deployment proposal for IP Anycast Evaluating the
Effects of Anycast on DNS Root Nameservers Measuring Query Latency of
Top Level DNS Servers Anycast vs.~DDoS: Evaluating the November 2015
Root DNS Event On the Use of Anycast in DNS (studies looking at anycast
in context of CDNs) fastroute A Practical Architecture for an Anycast
CDN Edgecast paper that hasn't been released yet Analyzing the
Performance of an Anycast CDN (studies looking at recursive
resolvers/caching) (maybe) John's TR of how different resolvers query at
different times On Modern DNS Behavior and Properties DNS Performance
and the Effectiveness of Caching D-mystifying the D-root Address Change
Authority Server Selection of DNS Caching Resolvers{[}d{]} Recursives in
the Wild: Engineering Authoritative DNS Servers (studies looking at web
performance/how user caching effects it) Measuring and mitigating web
performance bottlenecks in broadband access networks WePR: A tool for
Automated Web Performance Measurement Demystifying Page Load Performance
with WProf{[}e{]} Practical Challenge Response for DNS DNS Resolvers
Considered Harmful{[}f{]} Studies looking at root servers and queries
that land at them On eliminating root nameservers from the DNS DNS
Measurements at a Root Server A Day at the Root of the Internet

\fi

\section{Conclusion}\label{conclusion-1}

IP anycast has come under attack, with studies showing, for example, how
BGP can naturally route users to suboptimal anycast instances and
inflate user latencies. Due to the relative availability of root DNS
data and diverse deployment strategies of the root DNS servers, they are
common targets for delineating inefficiencies and suggesting
improvements to IP anycast. We argue not only that the root DNS servers
have different design goals (i.e.~resiliency against attacks) than that
of other anycast services, but also that users rarely interact with the
root DNS infrastructure -- rendering perceived inefficiencies and
proposed improvements to be ill-founded when only tested on the root
DNS. Perhaps simple yet effective ideas such as browser link prefetching
or DNS request parallelization should be expanded and their adoption by
users encouraged, rather than proposed improvements to IP anycast.
{[}a{]}ideally we would control the linebreak in the latex so it occurs
after ``it,'' {[}b{]}this argument about prior studies needs to be more
careful, I think. I propose an alternate abstract below {[}c{]}might be
an unfair comparison since the root DNS has many more nodes than
Microsoft {[}d{]}interesting that in 10 minutes they observe so many
queries for COM TLD yet don't see any issue with that {[}e{]}Might be an
interesting tool to use {[}f{]}Mark shared in an email -- shows time
between DNS queries \& TCP connection starts can be big, which suggests
DNS is not blocking

\bibliography{bib.bib}

\end{document}
