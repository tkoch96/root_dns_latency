\documentclass[sigconf,letterpaper,nonacm,10pt]{acmart}
\usepackage{lmodern}
\usepackage{amssymb,amsmath}
\usepackage{ifxetex,ifluatex}
\usepackage{fixltx2e} % provides \textsubscript
\ifnum 0\ifxetex 1\fi\ifluatex 1\fi=0 % if pdftex
  \usepackage[T1]{fontenc}
  \usepackage[utf8]{inputenc}
\else % if luatex or xelatex
  \ifxetex
    \usepackage{mathspec}
  \else
    \usepackage{fontspec}
  \fi
  \defaultfontfeatures{Ligatures=TeX,Scale=MatchLowercase}
  \newcommand{\euro}{€}
\fi
% use upquote if available, for straight quotes in verbatim environments
\IfFileExists{upquote.sty}{\usepackage{upquote}}{}
% use microtype if available
\IfFileExists{microtype.sty}{%
\usepackage{microtype}
\UseMicrotypeSet[protrusion]{basicmath} % disable protrusion for tt fonts
}{}
\usepackage{hyperref}
\PassOptionsToPackage{usenames,dvipsnames}{color} % color is loaded by hyperref
\hypersetup{unicode=true,
            pdftitle={If a Path is Inflated, and Noone Uses It, Is It Inefficient{[}a{]}?},
            colorlinks=true,
            linkcolor=blue,
            citecolor=blue,
            anchorcolor=blue,
            urlcolor=blue,
            breaklinks=true}
\urlstyle{same}  % don't use monospace font for urls
\usepackage{natbib}
\bibliographystyle{ACM-Reference-Format.bst}
\usepackage{graphicx,grffile}
\makeatletter
\def\maxwidth{\ifdim\Gin@nat@width>\linewidth\linewidth\else\Gin@nat@width\fi}
\def\maxheight{\ifdim\Gin@nat@height>\textheight\textheight\else\Gin@nat@height\fi}
\makeatother
% Scale images if necessary, so that they will not overflow the page
% margins by default, and it is still possible to overwrite the defaults
% using explicit options in \includegraphics[width, height, ...]{}
\setkeys{Gin}{width=\maxwidth,height=\maxheight,keepaspectratio}
\setlength{\emergencystretch}{3em}  % prevent overfull lines
\providecommand{\tightlist}{%
  \setlength{\itemsep}{0pt}\setlength{\parskip}{0pt}}
\setcounter{secnumdepth}{5}

%\usepackage{caption}
%\renewcommand{\captionfont}{\small} %small fonts for caption
%\renewcommand{\captionlabelfont}{\small}

% \usepackage{url}
% \usepackage{balance}

%% Adding URL breaks
% \makeatletter
% \g@addto@macro{\UrlBreaks}{\UrlOrds}
% \makeatother

% \usepackage{lastpage}

%\usepackage[aboveskip=2pt]{subcaption} %for subfigures

%\setlength{\textfloatsep}{0pt} %spacing between figures and texts
%\setlength{\floatsep}{0pt} 
%\setlength{\dblfloatsep}{0pt}
%\setlength{\dbltextfloatsep}{0pt}
%\setlength{\abovecaptionskip}{0pt}
%\renewcommand{\footnotesize}{\scriptsize}

%\usepackage{etoolbox} % spacing between formula and text
%\apptocmd\normalsize{%
%\abovedisplayskip=0pt
%\abovedisplayshortskip=0pt
%\belowdisplayskip=0pt
%\belowdisplayshortskip=0pt
%}{}{}

%\let\oldfootnote\footnote %small footnote
%\renewcommand{\footnote}[1]{{\oldfootnote{\scriptsize #1}}}


\usepackage{multirow}
\usepackage{graphicx}
%% conference information
% \acmYear{2019}
% \copyrightyear{2019}
% \acmConference{CoNEXT '19}{December 9-12, 2019}{Orlando, Florida, USA}

\hyphenation{name-spaces}

%% System name
\newcommand\thepeering{\textsc{Peering}\xspace}
\newcommand\peering{\textsc{Peering}\xspace}
%\newcommand\peering{\textsc{BGPPlatform}\xspace}
\newcommand\vbgp{\texttt{vBGP}\xspace}
\newcommand\testbed{platform\xspace}
\newcommand\testbeds{platforms\xspace}

%%% Macros for convenience
%% References
\newcommand{\secref}[1]{\S\ref{sec:#1}}
\newcommand{\figref}[1]{Figure~\ref{fig:#1}}
\newcommand{\tabref}[1]{Table~\ref{tab:#1}}
%% to refer to lines in graphs
\newcommand{\linename}[1]{\emph{#1}\xspace}
\newcommand{\parab}[1]{\smallskip\noindent {\bf #1}}

%% Common latin terms
\newcommand{\etc}{\emph{etc.}\xspace}
\newcommand{\ie}{\emph{i.e.,}\xspace}
\newcommand{\eg}{\emph{e.g.,}\xspace}
\newcommand{\etal}{\emph{et al.}\xspace}
\newcommand{\aka}{a.k.a\xspace}

%% editing notes
%\newcommand\ekb[1]{{\color{blue}[ekb: #1]}}
\newcommand\tbd[1]{{\color{red}{\bf TBD: #1}}}
%\newcommand\new[1]{#1}
%\newcommand\cut[1]{}
\definecolor{orange}{rgb}{1.0, 0.31, 0.0}
%\newcommand\edit[2]{#2}
%\newcommand\new[1]{{#1}}
%\newcommand\cut[1]{{#1}}

% terms
\newcommand\noescape[1]{#1}

\newcommand{\rightdownarrow}{\mathrel{\scalebox{1}[-1]{$\Rsh$}}}
\newcommand{\cmark}{\color{green}\ding{51}}
\newcommand{\xmark}{\color{red}\ding{55}}
\newcommand{\metroas}{$\left<\texttt{metro, AS, region}\right>$\xspace}
\newcommand{\fe}{front-end\xspace}
\newcommand{\feplural}{front-ends\xspace}
\newcommand{\capfe}{Front-end\xspace}
\newcommand{\capfeplural}{Front-ends\xspace}

\NewDocumentCommand{\rotseventy}{O{70} O{1em} m}{\makebox[#2][l]{\rotatebox{#1}{#3}}}


% De-Anonymized Commands
\iffalse
\newcommand\ISI{\textsc{the Information Sciences Institute (ISI) at USC}\xspace}
\fi

% Anonymized Commands
\newcommand\ISIone{a research lab in a university in the United States\xspace} % introductory reference
\newcommand\ISItwo{the research lab\xspace} % more general references
\newcommand\ISIthree{a research lab\xspace}

\title{If a Path is Inflated, and Noone Uses It, Is It Inefficient{[}a{]}?}
\author{
            Anonymized
        }
\date{}
\pagestyle{plain}

\begin{document}
\maketitle

\iffalse

\begin{itemize}
\tightlist
\item
  name: Thomas Koch affiliation: Columbia University
\item
  name: Matt Calder affiliation: Microsoft Research
\item
  name: Ethan Katz-Bassett affiliation: Columbia University
\item
  name: John Heidemann affiliation: ISI
\item
  name: Arpit Gupta affiliation: UCSB
\end{itemize}

Github: https://github.com/tkoch96/root\_dns\_latency Table of contents

\hypertarget{abstract}{%
\section*{Abstract}\label{abstract}}
\addcontentsline{toc}{section}{Abstract}

\hypertarget{introduction}{%
\section{Introduction}\label{introduction}}

\hypertarget{dns-a-practical-viewpoint}{%
\section{DNS -- A Practical Viewpoint}\label{dns-a-practical-viewpoint}}

\hypertarget{great-latency-savings-potential}{%
\subsection{Great Latency Savings
Potential}\label{great-latency-savings-potential}}

\hypertarget{practical-dns-pitfalls}{%
\subsection{Practical DNS Pitfalls}\label{practical-dns-pitfalls}}

\hypertarget{a-close-look-at-a-recursive-resolver}{%
\section{A Close Look at a Recursive
Resolver}\label{a-close-look-at-a-recursive-resolver}}

\hypertarget{a-global-look-at-users-of-a-large-cdn}{%
\section{A Global Look at Users of a Large
CDN}\label{a-global-look-at-users-of-a-large-cdn}}

\hypertarget{data-sources-and-processing}{%
\subsection{Data Sources and
Processing}\label{data-sources-and-processing}}

\hypertarget{root-latency-experienced-by-users}{%
\subsection{Root Latency Experienced by
Users}\label{root-latency-experienced-by-users}}

\hypertarget{related-work}{%
\section{Related Work}\label{related-work}}

\hypertarget{conclusion}{%
\section{Conclusion}\label{conclusion}}

Link to outline \fi

\hypertarget{abstract-1}{%
\section*{Abstract}\label{abstract-1}}
\addcontentsline{toc}{section}{Abstract}

Anycast is a means of distributing content that has been praised for its
simplicity and performance, yet criticized for inflating user latencies
in some cases. However, anycast CDNs and anycast DNS resolvers serve
latency-sensitive content to millions presenting an apparent
contradiction{[}b{]}{[}c{]}{[}d{]}. To resolve this discrepancy in
understanding, we analyze anycast in two important settings: the root
DNS and a large anycast CDN, placing an emphasis on how anycast's
performance translates to end user latency. First, we question whether
latency matters at all, and find that while it does in the CDN setting,
it makes little difference in the root DNS due to heavy caching near end
users -- root DNS resolution only accounts for at most 2 ms per day for
most users. We then examine how anycast specifically impacts performance
to end users, and again demonstrate users see negligible effects from
anycast path inflation. Users of a large anycast{[}e{]}{[}f{]}{[}g{]}
CDN, conversely, can experience a few ms per page load of anycast path
inflation, with the effect becoming more prevalent in larger anycast
deployments. The effects of the inflation, however, are dwarfed by the
performance benefits users see by, on average, visiting closer anycast
sites{[}h{]}{[}i{]}. Hence we demonstrate that although anycast can
inflate latencies, the degree to which this affects users largely
depends on the setting, and the prevalence of this inflation does not
tell the whole stor{[}j{]}y -- anycast still does a good job routing
users to sites.

\hypertarget{introduction-1}{%
\section{Introduction}\label{introduction-1}}

\label{sec:introduction} IP anycast, an approach to routing in which
geographically diverse servers known as anycast replicas all use the
same IP address, is used by a number of operational DNS
\cite{root_servers, cloudflare_anycast, akamai_anycast, route53_anycast, google_public_dns}
and CDN \cite{calder2015analyzing,edgecast_anycast,amazon_cloudfront}
systems today, in part because of its support to improve latency to
clients and decrease load on each anycast server
\cite{katabi2000framework,metz2002ip,rfc_1546}. However, studies
(\cite{sarat2006use, li_levin_spring_bhattacharjee_2018}) have argued
that anycast provides sub-optimal performance for some users, compared
to the lowest latency one could achieve given deployed replicas. Despite
results demonstrating theoretical routing inefficiency, how these
inefficiencies impact user experience is not well understood.{[}k{]}
Anycast, therefore, presents a paradox providing benefits of increased
capacity and decreased latency while also purportedly hurting
performance.

To resolve this, we take a step back and ask pointed questions -- why do
distributed systems such as CDNs and operational DNS use anycast, how
does anycast affect their performance, and how does this performance
ultimately translate to end user experience? Towards answering these
questions we analyze anycast from two different angles: the root DNS and
a large anycast CDN, chosen for their overlapping, yet distinctive,
goals. The root DNS servers feature in studies
\cite{colitti2006evaluating, moura2016anycast, de2017anycast, li_levin_spring_bhattacharjee_2018, mcquistin2019taming}
involving anycast because it is relatively easy to gain access to root
DNS data, because it is straightforward to gain access to information
about their deployments, and because they are run by several
organizations \cite{root_servers}. This last fact manifests itself in a
diverse set of deployment strategies, for essentially the same service.
Here, examining the root DNS and anycast CDNs is particularly
interesting because this analysis illustrates how the setting in which
we study anycast heavily influences the conclusions we can draw -- while
we find that mitigating anycast path inflation is quite important for
anycast CDNs, the impact of latency \textit{at all} in the root DNS
setting is negligible.

Although we do wish to examine how anycast affects performance both at
the roots and in anycast CDNs, we first take a step back and examine
whether performance (that is, latency) matters \textit{at all}
(\autoref{sec:root_dns_latency}). By leveraging global root DNS traces
and user data from a large anycast CDN, we demonstrate that the effect
of root DNS latency on user-perceived latency is negligible, accounting
for perhaps a few milliseconds of wait time per day or fractions of a
percent of a page load. This is due mostly to heavy caching of root DNS
records; hence, regardless of how inflated paths to the root replicas
are, this latency is amortized over large user populations. Conversely,
we show that latency matters considerably for anycast CDNs, comprising
tens of percents of page load time
(\autoref{sec:does_anycast_matter_cdn}). Collectively these results
suggest organizations managing the root servers have little incentive to
improve latency and, consequently, it makes little sense to evaluate the
latency achievable with anycast by looking at the root servers.

With these basic results about latency, we revisit frequently posed
questions about how anycast specifically impacts these services. We find
that, even though round-trip times differ significantly by root DNS
anycast deployment size, these differences are negligible when looked at
from a per-page load perspective, making at most an
INSERT\_NUMBER{[}l{]} millisecond difference. Similarly, even though we
find that increasing deployment size can lead to more inflation in the
roots{[}m{]}, this inflation negligibly factors into page load times
(\autoref{sec:root_dns_anycast}). Conversely, we find that for an
anycast CDN, although increasing deployment sizes does make anycast path
inflation more prevalent, the latency per page load decreases by tens of
milliseconds (\autoref{sec:cdn_anycast}) with additional sites.
Moreover, regardless of deployment size, the path inflation for the
large anycast CDN is less than half that of the roots for more than 90\%
of users (of the large anycast CDN). Hence larger deployment sizes can
provide tangible latency benefits to anycast CDNs, but probably provide
little benefit in the root DNS setting, and the magnitude of these
benefits are largely dependent on deployment details.

\hypertarget{background-anycast-in-distributed-systems}{%
\section{Background: Anycast in Distributed
Systems}\label{background-anycast-in-distributed-systems}}

\label{sec:anycast_distributed_systems} IP anycast is a system in which
geographically diverse servers known as anycast replicas all use the
same IP address, and therefore rely on BGP's notion of propagating the
best route towards the destination. We discuss why distributed services
on the Internet may wish to use Anycast, and ultimately leverage this
knowledge in later sections to discuss why the root DNS and CDNs may use
IP anycast. The benefits we are about to mention are \textit{potential}
-- the realization of such benefits depends on the deployment details of
the system using IP anycast. We also provide some background on the two
services we investigate -- the root DNS and a large anycast CDN, as this
context will prove useful in later sections.

\hypertarget{potential-benefits-of-anycast}{%
\subsection{Potential Benefits of
Anycast}\label{potential-benefits-of-anycast}}

{[} why distributed systems might use anycast, different performance
benefits they might shoot for {]} IP anycast is first and foremost,
simple and scalable. Network managers offload the responsibility of
mapping users to sites to the network, and may add more sites by simply
advertising the address from more locations. IP anycast can provide low
latencies from users to destinations, since it relies on the network's
notion of ``best path''; practically, this is handled by the Border
Gateway Protocol (BGP). Anycast also offers resiliency in that, should
one site go offline, the network will handle routing users to different
sites that are still running. A final potential benefit offered by
anycast is load balancing. The idea is that users are spread out over
the network, and so they will roughly be routed to corresponding anycast
sites in different areas of the network \cite{metz2002ip}. Although
recent research suggests anycast may not naturally balance load as was
previously thought \cite{li_levin_spring_bhattacharjee_2018}, what is
more potentially useful is that anycast routes traffic predictably.
Prior work suggests that only a very small fraction of anycast paths are
unstable \cite{wei2017does}, and so network managers may provision in
advance for expected (if unbalanced) load.

\hypertarget{root-dns-anycast}{%
\subsection{Root DNS anycast}\label{root-dns-anycast}}

{[} brief description of dns resolution process, caching potential {]}
DNS is a fundamental service on the Internet that maps human readable
hostnames to IP addresses \cite{cloudflare_dns_tutorial, rfc_1035}. To
resolve a hostname, a user will send DNS requests in the form of UDP
packets to one or more recursive resolvers (RRs) provided by their
ISP\footnote{ The user can specify whatever RR they wish, but one can sensibly assume the typical user's RR is set by the ISP, broadcasted through DHCP. }.
The RR then requests the records from a root DNS server, top level
domain (TLD) server and authoritative DNS (ADNS) server corresponding to
the record the user requested. Since each request is a correspondence
between the RR and a remote server, there can be several requests made
by the RR for a single end-user request.

The root DNS servers \cite{root_servers} are grouped into thirteen
letters, and are managed by twelve distinct organizations. Each letter
consists of a certain number of anycast replicas, with actual numbers
ranging from a few to a few hundred, and each letter is assigned a
unique v4 and v6 address. Each replica serves DNS records for the TLD
records, of which there are a little over 1,000. All but two of these
records has a TTL of two days, with the exceptions having TTL's of one
day. When a RR needs to query a root server, it may query whichever one
it wishes (subject to network administration policy); however, recursive
resolver software is known to query high performing (that is, low
latency) letters more often. The performance of anycast in the root DNS
has been extensively studied in a variety of contexts
\cite{colitti2006evaluating, moura2016anycast, de2017anycast, li_levin_spring_bhattacharjee_2018, mcquistin2019taming}.

\hypertarget{cdn-anycast}{%
\subsection{CDN Anycast}\label{cdn-anycast}}

{[} description of microsoft's CDN, with a focus on intricacies of
rings, and how we use them to simulate deployments{]} To investigate
anycast{[}n{]}, To study anycast in a setting that contrasts the root
DNS, we leverage a large anycast CDN that serves millions of users from
more than 100 front ends. Traffic destined for the large anycast CDN
will enter the CDNs network at a peering point, and be routed to one of
the anycast replicas serving the content. This large anycast CDN has a
logical hierarchy of layers, called rings, that serve different types of
content. Hence, traffic from a user prefix destined for the CDN may end
up at two different front ends (depending on the service), but will
ingress into the network at the same peering point.

In the following, we use these rings to simulate anycast deployments of
different sizes, as a way of coarsely assessing anycast's behavior with
increasing deployment size. That is, since various services use
different sized rings, we are able to isolate performance metrics for
each ring. Although the front ends serving content are anycasted, this
is not a perfect analogy to actually deploying different sized anycast
networks on the Internet, since traffic traversing the public Internet
will take the same path, regardless of the ring in question. Hence we
are, to some extent, only assessing anycast within the CDNs network.

\hypertarget{does-dns-root-latency-matter}{%
\section{Does DNS Root Latency
Matter?}\label{does-dns-root-latency-matter}}

\label{sec:root_dns_latency} Before answering questions about anycast
latency, we would first like to understand root DNS latency. Here we not
only measure what root DNS latencies are{[}o{]}, but also how this
ground truth latency manifests itself in \textit{user perceived}
latency. We approach this from two main perspectives: local and global.
The former allows us to estimate the fraction of page load time (PLT)
during which a client is waiting for root DNS resolution and the latter
allows us to estimate the total time per day a user is waiting for root
DNS resolution.

{[}ISI results -- root dns latency cdf and PLT implications Main idea is
cache hit rate is high, and root requests are rarely generated. {]}

{[}DITL -- root DNS per day latency; Main idea is that amortizing
requests seen to the roots over large user populations makes latency
implications small{]}

{[}Embedded in the above is a comparison between high latency and low
latency roots with implications to user latency, so No, latency doesn't
matter{]}

\hypertarget{anycast-performance-in-the-root-dns}{%
\section{Anycast Performance in the Root
DNS}\label{anycast-performance-in-the-root-dns}}

\label{sec:root_dns_anycast} {[}anycast path inflation per RTT \& per
page load; Main idea is path inflation does become more prevalent with
increasing deployment, but makes no difference to users{]} {[}bar graph
showing which sites are hit, corroborates the idea that inefficiency
grows with deployment size, usually{]}

{[}use these results to suggest sites are added to root deployments for
resilience, since there is no difference from a PPL perspective{]}

\hypertarget{does-anycast-cdn-latency-matter}{%
\section{Does Anycast CDN Latency
Matter?}\label{does-anycast-cdn-latency-matter}}

\label{sec:does_anycast_matter_cdn} {[}latency per RTT and per page load
at various ring sizes; Main idea is RTT latency (and perceived latency)
is significant from a PPL perspective, and ?depends on the
application?{]} {[}compared to root DNS per page load; idea is CDN
latency is orders of magnitude more significant PPL, and latency PPL
10th percentile vs 90th percentile is large{]}

\hypertarget{anycast-performance-in-cdnsp}{%
\section{Anycast Performance in
CDNs{[}p{]}}\label{anycast-performance-in-cdnsp}}

\label{sec:cdn_anycast} {[}path inflation per RTT/page load, by ring;
Main idea is that inflation becomes more prevalent, but latency PPL
still goes down{]} {[}inefficiency by ring; shows fewer users go to the
closest site, but still latency PPL goes down{]} {[}geographic path
inflation per RTT/page load, compared to the roots; CDN inflation
\textless{} root inflation,{[}q{]} argue that CDN works to control it
via peering{[}r{]}{]} {[}case studies of intermetro variability, or
unexpectedly poor performance, highlighting the intricacies (time
permitting){]}

\hypertarget{related-work-1}{%
\section{Related Work}\label{related-work-1}}

\label{sec:related} IP anycast performance is usually studied in the
context of two applications: the root DNS servers, and CDNs. In addition
to these topics, we discuss studies of popular recursive resolvers, and
user-centric measurements of web performance.

\hypertarget{root-dns-anycast-performance}{%
\subsection{Root DNS Anycast
Performance}\label{root-dns-anycast-performance}}

The performance of anycast in the context of root DNS is generally
gauged by anycast's ability to balance load among server replicas or
provide low latency to users. Generally, all studies conclude that
anycast successfully balances load, while latency performance depends on
the specific deployment configuration. \cite{moura2016anycast} looks at
a DDoS attack on the root name server infrastructure, and generally
shows that anycast is a good defense mechanism against such attacks. An
earlier study, \cite{sarat2006use} confirms that anycast protects the
root DNS infrastructure against such attacks and, furthermore, that
anycast routes users to an optimal location in most cases.
\cite{de2017anycast} looks at user latency to C, F, K, and L-root and
attributes better performance to good geographic location and peering
strategies. These findings coincide with an earlier study,
\cite{ballani2006measurement}, who conclude the performance of anycast
is intrinsically linked to deployment strategy. Additionally
\cite{de2017anycast} finds that as few as 12 sites can provide ``good''
latency to users. \cite{li_levin_spring_bhattacharjee_2018},
\cite{colitti2006evaluating}, \cite{de2017anycast}, and
\cite{liang2013measuring} are all examples of studies who quantify
latencies to various root servers, and note how these compare to the
(optimal) latency of the closest unicast alternative for the user who
issued the query.

\hypertarget{cdn-anycast-performance}{%
\subsection{CDN Anycast Performance}\label{cdn-anycast-performance}}

Some CDNs (e.g.~Cloudflare, Edgecast, Fastly) use IP anycast to augment
their serving infrastructure. When deploying an Anycast CDN (ACDN),
delivering content to users with low latency becomes a high priority, as
there is a large financial incentive to do so. The simplicity of IP
anycast comes at the cost of having coarse grained control over where
user queries land. Shifting user load between nodes during peak hours,
for example, is a challenging problem. As a potential solution,
\cite{alzoubi2011practical} and \cite{flavel2015fastroute} use DNS
redirects at ADNS servers to shift load among anycast nodes, albeit in
slightly different ways. \cite{calder2015analyzing} analyzes what
latency users are achieving, compared to optimal, when being routed to
anycast nodes and finds that 10\% of users experience a latency
inflation of at least 100 ms.

\hypertarget{recursive-resolvers-and-the-benefits-of-caching}{%
\subsection{Recursive Resolvers and the Benefits of
Caching}\label{recursive-resolvers-and-the-benefits-of-caching}}

Similar to the RR analysis conducted here, \cite{jung2002dns} looks at
DNS traffic on a small network and notably finds that 16\% of queries
resulted in queries to the root, most of which were for invalid domains.
As this study is quite old, it is no surprise that this rate has
decreased (recall we observed .5\% of queries resulted in queries to the
root) since browser designers and network engineers understand the
importance of caching. \cite{callahan2013modern} also looks at a RR and
analyzes statistics of DNS exchanges occurring over it including DNS
transaction latencies. Both \cite{yu2012authority} and \cite{lentz2013d}
look at certain pathological behaviors of popular recursive resolvers,
and the implications these behaviors have on root DNS load.

\hypertarget{web-performance}{%
\subsection{Web Performance}\label{web-performance}}

Although we were unable to find any specific study that looked at how
web performance and root DNS latency were related, there are certainly
studies characterizing web performance. \cite{sundaresan2013web}
characterizes web performance bottlenecks in (at the time) new broadband
networks, and finds that latency is the main bottleneck for PLT when the
user's bandwidth exceeds 16 Mbps. However, the study does not
realistically emulate a page load and, in particular, can not analyze
the effect of having multiple DNS resolutions per page. Similarly,
\cite{asrese2016wepr} analyzes how each step of a page load contributes
to the aggregate PLT using a tool designed in-house. However, unlike
\cite{sundaresan2013web}, they did not conduct a large measurement
campaign and do not include information about multiple DNS lookups per
page. A more recent study, \cite{enghardt2019web} provides a brief
survey of web performance measurement studies and explains why it is
difficult (with current practices) to compare two different studies in
web performance.

\iffalse

(studies looking at anycast in context of root DNS servers) Anycast
Performance, Problems and Potential how many sites are enough? A
measurement based deployment proposal for IP Anycast Evaluating the
Effects of Anycast on DNS Root Nameservers Measuring Query Latency of
Top Level DNS Servers Anycast vs.~DDoS: Evaluating the November 2015
Root DNS Event On the Use of Anycast in DNS (studies looking at anycast
in context of CDNs) fastroute A Practical Architecture for an Anycast
CDN Edgecast paper that hasn't been released yet Analyzing the
Performance of an Anycast CDN (studies looking at recursive
resolvers/caching) (maybe) John's TR of how different resolvers query at
different times On Modern DNS Behavior and Properties DNS Performance
and the Effectiveness of Caching D-mystifying the D-root Address Change
Authority Server Selection of DNS Caching Resolvers{[}s{]} Recursives in
the Wild: Engineering Authoritative DNS Servers (studies looking at web
performance/how user caching effects it) Measuring and mitigating web
performance bottlenecks in broadband access networks WePR: A tool for
Automated Web Performance Measurement Demystifying Page Load Performance
with WProf{[}t{]} Practical Challenge Response for DNS DNS Resolvers
Considered Harmful{[}u{]} Studies looking at root servers and queries
that land at them On eliminating root nameservers from the DNS DNS
Measurements at a Root Server A Day at the Root of the Internet

\fi

\hypertarget{conclusion-1}{%
\section{Conclusion}\label{conclusion-1}}

{[}a{]}ideally we would control the linebreak in the latex so it occurs
after ``it,'' {[}b{]}I'm not sure it is a contradiction. Aren't the
people criticizing it well aware that it serves many users? {[}c{]}I was
trying to get across that it is odd for (for example) Microsoft to use
anycast to serve latency sensitive content if it is known Anycast
inflates latencies. {[}d{]}Yes, I think your instinct of wanting to
convey that is good, I was just suggesting thinking a bit more about the
phrasing. {[}e{]}What is ``the'' CDN? {[}f{]}I mean to change this to a
macro, switching between ``a large anycast CDN'' and ``Microsoft's
anycast CDN''.

Do you think I need to talk more about what sort of data I use to draw
these conclusions in the abstract? {[}g{]}You might consider a brief
phrase like ``Analyzing global traces gathered from a large anycast CDN
and from all root DNS servers worldwide''. Not necessary in the
abstract, but can help. In the intro, you definitely want to talk a
little about your data. {[}h{]}It isn't clear what this is referring to,
closer than what, etc {[}i{]}Since these results and what we want to say
about them are in flux, I will wait before clarifying. {[}j{]}Will we
also have an argument that it is less prevalent for CDNs? The analysis
is in flux, and the data isn't perfect. {[}k{]}This seems important but
is buried in the middle of a paragraph {[}l{]}I recommend not putting
any placeholders in without markup. We should use markup like \tbd or
\todo that makes it colored, bold, and perhaps renders as something like
{[}{[}{[}TODO: INSERT NUMBER{]}{]}{]} {[}m{]}you use this phrase a lot
in the abstract and intro, but I'm not sure it's clear what you mean
{[}n{]}+mcalder@microsoft.com Is this a valid description of Microsoft's
anycast network, in the context that we leverage it? Please let me know
if you think any major details are missing. {[}o{]}perhaps a quick set
of speed checker measurements {[}p{]}this analysis is in flux
{[}q{]}might be an unfair comparison since the root DNS has many more
nodes than Microsoft {[}r{]}and more active debugging of bad routes,
probably {[}s{]}interesting that in 10 minutes they observe so many
queries for COM TLD yet don't see any issue with that {[}t{]}Might be an
interesting tool to use {[}u{]}Mark shared in an email -- shows time
between DNS queries \& TCP connection starts can be big, which suggests
DNS is not blocking

\bibliography{bib.bib}

\end{document}
