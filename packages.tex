\usepackage{xspace} % auto-eats spaces as needed
\usepackage{verbatim} % to support to-dos / comment environment
\usepackage{url} % links to bib items
% \usepackage{amsmath, amssymb} % math formulas, etc.
\usepackage{graphicx} % PDF, EPS, and many other graphics formats
\usepackage{multirow} % easier tables
% \usepackage{subfig}
\usepackage{float}
\usepackage[nolist]{acronym}
\usepackage{wasysym}
\usepackage[svgnames,x11names]{xcolor}
\usepackage{tabularx}
\usepackage{xparse}
\usepackage{pifont}
\PassOptionsToPackage{usenames,dvipsnames}{xcolor}
\usepackage{balance}
\usepackage{soul}
\definecolor{lightlightgray}{rgb}{.9, .9, .9}

\usepackage{cleveref}
\crefformat{section}{#2\S#1#3}
\Crefformat{section}{#2Section~#1#3}
\crefname{section}{\S}{\S\S}
\Crefname{section}{Section}{Sections}
\crefname{equation}{Eq.}{Eqs.}
\Crefname{equation}{Equation}{Equations}
\crefname{table}{Table}{Tables}
\Crefname{table}{Table}{Tables}
\crefname{figure}{Fig.}{Figs.}
\Crefname{figure}{Figure}{Figures}

%\usepackage{authblk}
%
%\makeatletter
%\renewcommand\AB@affilsepx{, \protect\Affilfont}
%\makeatother
%\renewcommand\Authands{, }

\def\UrlBreaks{\do\/\do-\do.}
\def\UrlNoBreaks{\do:}

% subheaders
\newtheorem{Theorem}{Theorem}
\newtheorem{Lemma}{Lemma}
\newtheorem{Problem}{Problem}
\newtheorem{Definition}{Definition}

% caption setup
\usepackage{caption}
\captionsetup[table]{position=bottom, skip=4pt}
\captionsetup[figure]{position=bottom, skip=1pt}
%\setlength{\textfloatsep}{0.7\baselineskip plus 0.2\baselineskip minus 0.5\baselineskip} 

% \usepackage{caption}
\setlength{\abovecaptionskip}{1pt}
\setlength{\belowcaptionskip}{-6pt}
% \captionsetup[table]{position=bottom, skip=4pt} 
\renewcommand{\captionfont}{\small}

% float parameters
% needed to create an algorithm float
% \usepackage{float}
% \renewcommand{\topfraction}{0.99}
% \renewcommand{\dbltopfraction}{0.99}
% \renewcommand{\bottomfraction}{0.99}
% \renewcommand{\floatpagefraction}{0.99}
% \renewcommand{\dblfloatpagefraction}{0.99}
% \setcounter{totalnumber}{99}
% \setcounter{topnumber}{99}
% \setcounter{dbltopnumber}{99}

\usepackage{subfig}

% to make tabulars easier
% from
% https://tex.stackexchange.com/questions/2441/how-to-add-a-forced-line-break-inside-a-table-cell
% v is one of t,c,b and h one of l,c,r
% \thead[bc]{blah \\ blah}
\usepackage{makecell}

